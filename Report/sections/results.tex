\chapter{Experiments and Results}
\label{sec:results}

% Evaluation of the pcfg prior:
% Show the pcfg model and how it prefers long notes on downbeats and how it penalises syncopation.
First we will train the rhythm and expression model on the entire jazz corpus like we described in section \ref{sec:training} and analyse the results. Second, we will use 10-fold cross validation to evaluate the performance of the parser on the corpus. We will compare the expression model presented here to a expression model that treats expression as additive noise. That is, for any feature vector $\mu_\varphi$ is set to zero and $\sigma_\varphi$ is set to a small value.

\section{Rhythm and Expression Model}

The rhythm and expression model, trained on the entire jazz corpus are shown in table \ref{tab:models}. 

\begin{table}
\caption{The rhythm and expression model, trained on the entire jazz corpus.}
\label{tab:models}
\centering
\subfloat[The rhythm model.]{
\label{tab:rhythm}
\begin{tabular}{ll}
\hline
\textbf{Rule}& \textbf{Probability}\\
\hline
\hline
$R \rightarrow \bullet\; \bullet$ & 0.061\\
$R \rightarrow *\; \bullet$ & 0.027\\
$R \rightarrow R\; R$ & 0.392\\
$R \rightarrow \bullet\; R$ & 0.057\\
$R \rightarrow R\; \bullet$ & 0.022\\
$R \rightarrow *\; R$ & 0.069\\
$R \rightarrow R\; *$ & 0.061\\
$R \rightarrow \bullet\; \bullet\; \bullet$ & 0.013\\
$R \rightarrow \bullet\; *\; \bullet$ & 0.164\\
$R \rightarrow *\; *\; \bullet$ & 0.134\\
\hline
\end{tabular}
}
\subfloat[The expression model.]{
\label{tab:expression}
\begin{tabular}{lll}
\hline
$\varphi = [\texttt{level}, \texttt{division}]$ & $\mu_\varphi$ & $\sigma_\varphi$\\
\hline
\hline
$[1, 2]$ & $7.926 x 10^{-2}$ & $3.391 x 10^{-2}$\\
$[1, 3]$ & $-6.323 x 10^{-2}$ & $0.656$\\
$[2, 2]$ & $-4.68 x 10^{-3}$ & $2.184 x 10^{-2}$\\
$[3, 2]$ & $5.565 x 10^{-3}$ & $9.422 x 10^{-3}$\\
$[4, 2]$ & $4.797 x 10^{-3}$ & $9.824 x 10^{-3}$\\
$[5, 2]$ & $-3.391 x 10^{-3}$ & $1.887 x 10^{-2}$\\
$[6, 2]$ & $-7.539 x 10^{-5}$ & $1.375 x 10^{-3}$\\
$[7, 2]$ & $-8.029 x 10^{-3}$ & $1.223 x 10^{-3}$\\
$[8, 2]$ & $0.0$ & $0.0$\\
$[9, 2]$ & $0.0$ & $0.0$\\
\hline
\end{tabular}
}
\end{table}

The small $\mu$ and $\sigma$ values at high level are a consequence of the way that the performers of our corpus recorded the melodies. The melodies were probably played along with some metronomic track so that the overall tempo does not change.

The expression model shows a slight stretching of downbeats in duple divided units at the lowest level. The values shown in \ref{tab:expression} are logarithmic ratios. Therefore, the average stretching of downbeats at level one can be found by taking the exponential: $\exp(7.926 x 10^{-2}) \approx 1.082$. 

For triple divisions at the lowest level, the results seem to be more noisy and show that on average, downbeats are played shorter. This could shows, perhaps not surprisingly [Explain, reference], that the swung notes are not consistently played in a 3:2 ratio.

At level two and above, the average logarithmic ratios seem to be fairly close to zero.

\section{Performance on the Corpus}

Since we do not have enough data to keep a separate test set, we will train the models on different trainsets and evaluate them on small parts of the corpus left out of the train set. This process is known as cross validation. For $n$-fold cross validation, the the corpus is divided into $n$ equal parts, a training set is constructed from $n-1$ of these parts and a test set is constructed of one of these parts. This is done $n$ times and the parts sampled randomly from the corpus. When dividing into $n$ random sampled parts, performances are treated as whole units. Since there are twenty different performances in the corpus, using 10-fold cross validation results in train sets of eighteen performances that will be evaluated on test sets of two performances.

It was mentioned in section \ref{sec:evaluation} that our implementation of the parser is not computationally efficient enough to parse whole performances, therefore we will only consider the first few notes of each performance in the test set. 

Our subdivision trees can only represent performances with a length, measured in whatever metrical unit, that is a power of two. If a performance is for example three bars long, the subdivision tree will represent a fourth bar as a tie. To make evaluation easier, the test performances are restricted to have lengths that are a power of two. This is done by selecting, given some preferred length, the leftmost nested subdivision tree that contains a number of onsets closest to the preferred length. We can find the subdivision tree in the labels of our corpus.

We ran two experiments. The expression model as it was described in section \ref{sec:method} was used in the first experiment. In the second experiment, non-zero expression ratios were simply treated as noise with a standard deviation of 0.1. The results of these experiments are shown in table \ref{tab:results_expression} and \ref{tab:results_noise}.

\begin{table}
\centering
\caption{10-fold cross validation on the jazz corpus for different functions and different tests sets.}
\label{tab:results_expression}
\begin{tabular}{llll}
\hline
\textbf{Preferred length} & \textbf{Precision} & \textbf{Recall} & \textbf{F-score}\\
\hline
\hline
5 & $0.533$ & $0.472$ & $0.5$\\
10 & $0.451$ & $0.497$ & $0.472$\\
15 & $0.516$ & $0.493$ & $0.504$\\
20 & $0.524$ & $0.54$ & $0.532$\\
\hline

\end{tabular}
\end{table}

\begin{table}
\centering
\caption{10-fold cross validation on the jazz corpus for different functions and different tests sets. The expression model is set to additive noise with $\mu = 0$ and $\sigma = 0.1$.}
\label{tab:results_noise}
\begin{tabular}{llll}
\hline
\textbf{Preferred length} & \textbf{Precision} & \textbf{Recall} & \textbf{F-score}\\
\hline
\hline
5 & $0.841$ & $0.765$ & $0.802$\\
10 & $0.866$ & $0.831$ & $0.848$\\
15 & $0.765$ & $0.75$ & $0.757$\\
20 & $0.83$ & $0.816$ & $0.823$\\
\hline

\end{tabular}
\end{table}

These results clearly show that the expression model performs worse than the model that treats expression simply as noise. Section \ref{sec:discussion} will discuss what may have caused the low results for the expression model.