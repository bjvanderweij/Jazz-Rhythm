\chapter{Experiments and Results}
\label{sec:results}

% Evaluation of the pcfg prior:
% Show the pcfg model and how it prefers long notes on downbeats and how it penalises syncopation.
First, we will train the rhythm and expression model on the entire jazz corpus as we described in section \ref{sec:training} and analyse the results. Second, we will use 10-fold cross validation to evaluate the performance of the parser on the corpus. We will compare the expression model presented here to an expression model that treats expression as additive noise. That is, for any feature vector $\mu_\varphi$ is set to zero and $\sigma_\varphi$ is set to a small value.

\section{Rhythm and Expression Model}

The rhythm and expression model, trained on the entire jazz corpus are shown in table \ref{tab:models}. 

\begin{table}
\caption{The rhythm and expression model, trained on the entire jazz corpus.}
\label{tab:models}
\centering
\subfloat[The rhythm model.]{
\label{tab:rhythm}
\begin{tabular}{ll}
\hline
\textbf{Rule}& \textbf{Probability}\\
\hline
\hline
$R \rightarrow \bullet\; \bullet$ & 0.061\\
$R \rightarrow *\; \bullet$ & 0.027\\
$R \rightarrow R\; R$ & 0.392\\
$R \rightarrow \bullet\; R$ & 0.057\\
$R \rightarrow R\; \bullet$ & 0.022\\
$R \rightarrow *\; R$ & 0.069\\
$R \rightarrow R\; *$ & 0.061\\
$R \rightarrow \bullet\; \bullet\; \bullet$ & 0.013\\
$R \rightarrow \bullet\; *\; \bullet$ & 0.164\\
$R \rightarrow *\; *\; \bullet$ & 0.134\\
\hline
\end{tabular}
}
\subfloat[The expression model.]{
\label{tab:expression}
\begin{tabular}{lll}
\hline
$\varphi = [\texttt{level}, \texttt{division}]$ & $\mu_\varphi$ & $\sigma_\varphi$\\
\hline
\hline
$[1, 2]$ & $7.926 x 10^{-2}$ & $3.391 x 10^{-2}$\\
$[1, 3]$ & $-6.323 x 10^{-2}$ & $0.656$\\
$[2, 2]$ & $-4.68 x 10^{-3}$ & $2.184 x 10^{-2}$\\
$[3, 2]$ & $5.565 x 10^{-3}$ & $9.422 x 10^{-3}$\\
$[4, 2]$ & $4.797 x 10^{-3}$ & $9.824 x 10^{-3}$\\
$[5, 2]$ & $-3.391 x 10^{-3}$ & $1.887 x 10^{-2}$\\
$[6, 2]$ & $-7.539 x 10^{-5}$ & $1.375 x 10^{-3}$\\
$[7, 2]$ & $-8.029 x 10^{-3}$ & $1.223 x 10^{-3}$\\
$[8, 2]$ & $0.0$ & $0.0$\\
$[9, 2]$ & $0.0$ & $0.0$\\
\hline
\end{tabular}
}
\end{table}

The expression model shows a slight stretching of downbeats in duple divided units at the lowest level. The values shown in \ref{tab:expression} are logarithmic ratios. Therefore, the average stretching of downbeats at level one can be found by taking the exponential: $\exp(7.926 x 10^{-2}) \approx 1.082$. 

The small $\mu$ and $\sigma$ values at higher levels are a consequence of the way that the performers of our corpus recorded the melodies. As we have said earlier, the expression ratio reflects tempo changes at higher levels. The melodies in the jazz corpus were often played along with a metronomic accompaniment track so that the global tempo does not change. This results in near-zero expression ratios at high levels.

\section{Performance on the Corpus}

We prepared three experiments which we tested on four different preferred lengths: 5, 10, 15 and 20 (see chapter \ref{sec:evaluation}). In the first experiment we used our PCFG rhythm model in combination with our expression model, the results are shown in table \ref{tab:results1}. In the second experiment, we used our own PCFG rhythm model combined with the alternative expression model that treats non-zero expression ratios as noise with a standard deviation of 0.1. The results of this experiment are shown in table \ref{tab:results2}. The final experiment served as a baseline: a rhythm prior that assigns the same probability to every rhythm was used in combination with the expression model that treats expression as noise. The results of the final experiment are shown in table \ref{tab:results3}.

\pagebreak
\begin{table}
\centering
\caption{Results for the expression model with the PCFG prior.}
\label{tab:results1}
\begin{tabular}{llll}
\hline
\textbf{Preferred length} & \textbf{Precision} & \textbf{Recall} & \textbf{F-score}\\
\hline
\hline
%5 & $0.533$ & $0.472$ & $0.500$\\
%10 & $0.451$ & $0.497$ & $0.472$\\
%15 & $0.516$ & $0.493$ & $0.504$\\
%20 & $0.524$ & $0.540$ & $0.532$\\
5 & 0.486 & 0.464 & 0.475\\
10 & 0.484 & 0.530 & 0.506\\
15 & 0.517 & 0.568 & 0.541\\
20 & 0.498 & 0.525 & 0.511\\
\hline
\end{tabular}
\end{table}

\begin{table}
\centering
\caption{Results for the model that treats expression as additive noise with $\mu = 0$ and $\sigma = 0.1$ and the PCFG prior.}
\label{tab:results2}
\begin{tabular}{llll}
\hline
\textbf{Preferred length} & \textbf{Precision} & \textbf{Recall} & \textbf{F-score}\\
\hline
\hline
%5 & $0.841$ & $0.765$ & $0.802$\\
%10 & $0.866$ & $0.831$ & $0.848$\\
%15 & $0.765$ & $0.750$ & $0.757$\\
%20 & $0.830$ & $0.816$ & $0.823$\\
5 & 0.842 & 0.789 & 0.815\\
10 & 0.718 & 0.692 & 0.705\\
15 & 0.850 & 0.831 & 0.840\\
20 & 0.682 & 0.696 & 0.689\\
\hline
\end{tabular}
\end{table}

\begin{table}
\centering
\caption{Baseline results. Expression is treated as additive noise and the prior returns the same probability for every rhythm.}
\label{tab:results3}
\begin{tabular}{llll}
\hline
\textbf{Preferred length} & \textbf{Precision} & \textbf{Recall} & \textbf{F-score}\\
\hline
\hline
%5 & 0.604 & 0.653 & 0.627\\
%10 & 0.692 & 0.745 & 0.718\\
%15 & 0.604 & 0.653 & 0.627\\
%20 & 0.513 & 0.584 & 0.546\\
5 & 0.378 & 0.518 & 0.437\\
10 & 0.410 & 0.55 & 0.470\\
15 & 0.396 & 0.472 & 0.431\\
20 & 0.389 & 0.471 & 0.426\\
\hline
\end{tabular}
\end{table}